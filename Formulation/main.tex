\documentclass{article}
\usepackage{graphicx} % Required for inserting images

\title{SPRO4 PROBLEM FORMULATION}
\author{Group 7}
\date{March 2023}

\begin{document}

\maketitle
 
\section{Background}

Possessing the ability of flight and minimising effort and casualties has always been desirable for the utility flight can provide. The first unmanned aircrafts can be dated back to 1849, where Austria seemingly had utilised unmanned air balloons with stuffed explosives to attack Venice. [1] Ever since an unmanned aircraft vehicle (UAV), is one that is flown by technological means or as a pre-programmed flight without pilot control, as defined by the ECAA Transport Agency [2], nowadays called drones, have risen in popularity.   


Because of this, UAVs come in a wide range of sizes and weights. UAVs often include multirotor, radio-controlled miniature helicopters, and aeroplanes [3]. As a result, there are several methods to categorize drones. The performance parameters of UAVs, such as weight, wingspan, wing load, flight range, maximum flying altitude, speed, and production cost, are typically used to categorize them [4]. According to how the lift is produced, drones may also be divided into fixed-wing and rotating-wing types. According to the drone code category, the European Aviation Safety Agency (EASA) categorizes unmanned aircraft by weight. The EASA regulations for open categories, or drones without an EASA class designation, are summarized succinctly and simply in Table I [5]. 
 
\graphicspath{}
 Table I Classification and restrictions for non-EASA class drones [5] 

Self-built drones weighing up to 250 g, as described in Table 1, may be used without registration if the drone is a toy or the drone is not equipped with a camera, the remaining drones must be registered, and the pilot must pass examinations [5]. In this paper, self-built rotary drones with four wings or propellers are the objective, making weight-based classification suitable. 

When it comes to the state-of-the-art project, PULP-DroNet is a deep learning-powered visual navigation engine that enables autonomous navigation of a pocket-size quadrotor in a previously unseen environment. Thanks to PULP-DroNet the nano-drone can explore the environment, avoiding collisions also with dynamic obstacles, in complete autonomy -- no human operator, no ad-hoc external signals, and no remote laptop! This means that all the complex computations are done directly aboard the vehicle and very fast. The visual navigation engine is composed of both a software and a hardware part. [6] 

When it comes to the future, the simulated pollination of agricultural plants by means of nano copter can provide collecting and delivering pollen in the mode of automatic control. A design of nano copter for pollination can be made on the basis of innovative modification of existing model by its reprogramming with regard to its flight controller that is to be fully adapted to computer interface. The robotic system is offered specially for artificial pollination in conditions of greenhouses and minor agricultural enterprises. [7] 

 
\section{Problem statement}

The utility of smaller drones are immense, where it can be used in surveillance, toys and potentially to also be part of a swarm of drones. Although, there are smaller drones existing in the current market, we would like to challenge ourselves to build one ourselves, where certain goals ranging from functionality to budget are listed below. 


\subsection{Primary goals }
\begin{itemize}
    \item 
    Net maximum weight of the drone is 250 grams. Weight under 250 grams ensures it falls under A1 category in EU regulations. 1 
    \item
    Flight time of 20 seconds.
    \item
    Stress of the structural system should not exceed rupture point. System does not experience fracture. 
\end{itemize}


\subsection{Secondary goals}
\begin{itemize}
    \item 
    Flight time of minimum one minute. 
    \item
    Can land with acceleration less than 9.8 m/s2.
    \item
    Stress of the drone system should not exceed the yield point. System does not experience plastic deformation. 
    \item
    Drone is remote controllable. 
    \item
    Drone can fly in formation with another identical drone.
    \item
    Total production cost of the drone is under 500 DKK (Not including remote controller). 
    \item
    Drone can play audio.
\end{itemize}


 
\subsection{Constraints}

\begin{itemize}
    \item 
    Budget for entirety of project is 2000 DKK.  
    \item
    Time available to finish the project is 4 months.
    \item
    Drone should have a minimum hover time of 5 seconds. 
    \item
    Drone should be fully functional and able to take off again after landing.
    \item
    No use of flight controller software or unmanned vehicle Autopilot software Suite, capable of controlling autonomous vehicles. 
\end{itemize}
 

\section {Test Specifications} 
\subsection{Primary goals:}
\begin{itemize}
    \item
    To test this, the drone will be weighed with a scale of a precision on 0,1 grams.
    \item  
    In order to test the flight time, a stopwatch will be started from the moment the drone leaves the ground and is stopped as soon as it lands.
    \item
    This goal will be the tested trough FEM, ensuring that the chosen material for the drones body, will not rupture.
\end{itemize}
  
\subsection{Secondary goals}
\begin{itemize}
    \item
    This will be tested with the same method as primary goals tests point 2. 
    \item
    This will be tested with a mobile phone, recording the drones landing, using the drones position compared to the timestamp of the video.  
    \item
    This will be tested with the same goals as primary goals test point 3.  
    \item
    This will be tested by the possibility of sending wireless signals to the drone, with the drone reacting to those send signals.  
    \item
    This will be tested purely by ear, listening to the drones output.  
    \item
    This will be tested by mobile phone video, looking at the drones positions at given timestamps.  
    \item
    This will be tested trough summing the price for each single part, ensuring that it doesn’t exceed 500 DKK. 
\end{itemize}

\subsection{Constraints}
\begin{itemize}
    \item
    This will be done with the same method as the secondary goals test, though ensuring the project cost is over 2000 DKK. 
    \item
    To evaluate the time constraint point of the project, the goal fullfilments will be evaluated in the end of the project period. In the case that all primary goals are fulfilled, the constraint is succeeded.  
    \item
    This will be tested with a stopwatch, ensuring that the hover time is atleast 5 seconds. 
    \item
    This will be tested with making the drone take off right after a landing, making sure that the drone is fully operational at the second take-off. 
    \item
    This will fulfilled by not employing any of the aforementioned in the drone. 
\end{itemize}

\section{Sources}
\begin{enumerate}
    \item 
    “A Brief History of Drones: The Remote Controlled Unmanned Aerial Vehicles (UAVs)”, Kashyap Vyas, Jun 29, 2020. Available: 
    
    https://interestingengineering.com/innovation/a-brief-history-of-drones-the-remote-controlled-unmanned-aerial-vehicles-uavs'
    \item
    ECAA, [online] Available: https://www.ecaa.ee/et/lennundustehnika-ja-lennutegevus/mehitamata-ohusoidukid-sealhulgas-droonid  
    \item
    A. Cavoukian, "Privacy and Drones: Unmanned Aerial Vehicles", Privacy by Design Canada, vol. 1, no. 1, pp. 1-27, 2012.  
    \item
    M. Hassanalian and A. Abdelkefi, "Classifications applications and design challenges of drones: A review", Progress in Aerospace Sciences, vol. 91, pp. 99-131, November 2016.
    
    https://www.sciencedirect.com/science/article/abs/pii/S0376042116301348?via%3Dihub 
    
    \item
    Open Category - Civil Drones, [online] Available: 
    
    https://www.easa.europa.eu/domains/civil-drones-rpas/open-category-civil-drones.
    \item
    V. Niculescu, L. Lamberti, F. Conti, L. Benini and D. Palossi, "Improving Autonomous Nano-Drones Performance via Automated End-to-End Optimization and Deployment of DNNs," in IEEE Journal on Emerging and Selected Topics in Circuits and Systems, vol. 11, no. 4, pp. 548-562, Dec. 2021, doi: 10.1109/JETCAS.2021.3126259. 
    \item
    R. N. Abutalipov, Y. V. Bolgov and H. M. Senov, "Flowering plants pollination robotic system for greenhouses by means of nano copter (drone aircraft)," 2016 IEEE Conference on Quality Management, Transport and Information Security, Information Technologies (IT&MQ&IS), Nalchik, Russia, 2016, pp. 7-9, doi: 10.1109/ITMQIS.2016.7751907. 
\end{enumerate}


\end{document}
