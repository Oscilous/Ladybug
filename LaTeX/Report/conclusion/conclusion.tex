\setlength{\parskip}{0pt}
\section{Evaluation}
The evaluation of the achievement of the goals are provided below, 
where they are coloured accordingly.

\begin{center}

  \bigbreak

  \begin{tabular}{ c|c|c } 
      \color{ForestGreen}{Green} & \color{BrickRed}{Red} & \color{YellowOrange}{Orange} \\ 
      \hline
      Achieved & Not Achieved & Inconclusive \\  
  \end{tabular}

  \bigbreak
  \bigbreak
  \bigbreak
  \textbf{Primary Goals} 
  \bigbreak

  \begin{tabular}{| m{33em} |}
      \hline
      \color{ForestGreen}
      1. The net maximum weight of the drone is 250 grams. 
      Weight under 250 grams ensures it falls under A1 
      category in EU regulations. \\ 
      \hline
      Description:  Using a scale with a 0.1 precision, the drones total flight weight was found to be 60 grams.  \\ 
      \hline
      \hline
      \color{YellowOrange}
      2. The drone has a flight time of 20 seconds. \\ 
      \hline
      Description:  While the drones battery allows it to fly a total time of approximately 90 seconds at max power, the problems from a control perspective made 		      the flight time of 1 minute impossible.    \\ 
      \hline
      \hline
      \color{ForestGreen}
      3. Stress of the drones structural system should 
      not exceed rupture point. System does not experience 
      fracture. \\
      \hline
      Description:  As described in the design section, FEM analysis was performed that showed that the stresses was far from the yield strength of the used material. \\ 
      \hline
  \end{tabular}

  \bigbreak
  \textbf{Secondary Goals} 
  \bigbreak

  \begin{tabular}{| m{33em} |}
      \hline
      \color{BrickRed}
      1. The drone has a flight time of minimum one minute.  \\ 
      \hline
      Description:  The drone is unable to carry a battery large enough to achieve the wanted flight time. \\ 
      \hline
      \hline
      \color{YellowOrange}
      2. The drone can land with acceleration less than 9.8 
      m/s2. \\ 
      \hline
      Description:   As the group did not achieve stable flight for the drone, it was not possible for the group to measure the downward accelearation when landing. 
      Altough the drone is expected to lack thrust, it still would have the needed thrust to make the descend slower than a free fall.\\ 
      \hline
      \hline
      \color{ForestGreen}
      3. Stress of the drone system should not exceed the 
      yield point. System does not experience plastic 
      deformation. \\
      \hline
      Description:  As described in the primary goals, this goal was found achieved by FEM. \\ 
      \hline
      \hline
      \color{ForestGreen}
      4. The drone is remote controllable. \\
      \hline
      Description:  Since it was possible to control the drone (stop/start) trough a bluetooth connection to the drone, this goal was achieved. \\ 
      \hline
      \hline
      \color{BrickRed}
      5. The drone can play audio.\\
      \hline
      Description:   Since the drone was not able to carry the extra weight of a speaker or a buzzer, this goal was not achieved\\ 
      \hline
      \hline
      \color{BrickRed}
      6. The drone can fly in formation with another 
      identical drone. \\
      \hline
      Description:  The time was not available to create an identical drone, and thus the goal was not achieved. \\ 
      \hline
      \hline
      \color{ForestGreen}
      7. Total production cost of the drone is under 500 DKK 
      (Not including remote controller). \\
      \hline
      Description:  Trough budget the production cost of a single unit was found to be 371,13 Kr. (Excluding the pcb production cost), and thus the goal was achieved. \\ 
      \hline
  \end{tabular}

  \bigbreak
  \textbf{Constraints} 
  \bigbreak

  \begin{tabular}{| m{33em} |}
      \hline
      \color{ForestGreen}
      1. Budget for entirety of project is 2000 DKK.  \\ 
      \hline
      Description:  The group did not exceed the purchasing budget and therefore achived the goal.\\ 
      \hline
      \hline
      \color{YellowOrange}
      2. Time available to finish the project is 4 months. \\ 
      \hline
      Description:  Since the group did not achieve stable free flight of the drone, this goal is seen as inconclusive\\ 
      \hline
      \hline
      \color{BrickRed}
      3. The drone should have a minimum hover time of 5 
      seconds.\\
      \hline
      Description:   This goal is not achieved, since the group did not achieve stable flight and thus was not able to hover for 5 seconds\\ 
      \hline
      \hline
      \color{ForestGreen}
      4. The drone should be fully functional and able 
      to take off again after landing.\\
      \hline
      Description: When testing free flight, it became apparent that the drone can crash and still takeoff immediately after without damages. It is therefore fully 
      expected that the drone would also be able to take off after a safe landing (if it was able to hover correctly).\\ 
      \hline
      \hline
      \color{ForestGreen}
      5. No use of flight controller software or unmanned 
      vehicle Autopilot software Suite, capable of 
      controlling autonomous vehicles.\\
      \hline
      Description:  Since the program is derived from the created simulink software solely, no flight controller software is employed and thus the goal is acheived. \\ 
      \hline
  \end{tabular}

\end{center}    

\section{Conclusion}

In the project the group has achieved making a drone that has an assisted semi-stable hover. After adding components to the drone, it was clear that the initial weight estimations of components was off and that the drone therefore didn't have enough thrust to take off from the ground. 
It was however possible to make the drone have a semi-stable hover being suspended from a string in the ceiling, so that the drones motors only uses thrust to adjust its angles. When testing the drone without any restraints (testing a possible free flight), it became clear that a stable free flight was not possible due to the mentioned thrust issues. The group decided to use DC-motors for the drone as they were lightweight and had promising thrust values, altough it became clear that using BLDC-motors could had given the needed excess thrust with a low extra weight trade off. Here it is though noted that the electronics would need to be extended as new drivers would be needed. As a final conclusion it is observed that using a single sided milled PCB as the structural frame is a sub-optimal solution as the electronic connections are limitied in the given case. It is expected that creating a smaller, possibly multi-layered PCB and then creating a frame of another material would be the ideal solution, saving both weight and reducing the overall size of the drone further.
