\section{Problem statement}
The utility of smaller drones are immense, where it can be used in surveillance, toys and potentially to also be part of a swarm of drones. Although, there are smaller drones existing in the current market, we would like to challenge ourselves to build one ourselves, where certain goals ranging from functionality to budget are listed below.
\subsection{Primary goals }
\begin{itemize}
    \item
          Net maximum weight of the drone is 250 grams. Weight under 250 grams ensures it falls under A1 category in EU regulations. 1
    \item
          Flight time of 20 seconds.
    \item
          Stress of the structural system should not exceed rupture point. System does not experience fracture.
\end{itemize}


\subsection{Secondary goals}
\begin{itemize}
    \item
          Flight time of minimum one minute.
    \item
          Can land with acceleration less than 9.8 m/s2.
    \item
          Stress of the drone system should not exceed the yield point. System does not experience plastic deformation.
    \item
          Drone is remote controllable.
    \item
          Drone can fly in formation with another identical drone.
    \item
          Total production cost of the drone is under 500 DKK (Not including remote controller).
    \item
          Drone can play audio.
\end{itemize}



\subsection{Constraints}

\begin{itemize}
    \item
          Budget for entirety of project is 2000 DKK.
    \item
          Time available to finish the project is 4 months.
    \item
          Drone should have a minimum hover time of 5 seconds.
    \item
          Drone should be fully functional and able to take off again after landing.
    \item
          No use of flight controller software or unmanned vehicle Autopilot software Suite, capable of controlling autonomous vehicles.
\end{itemize}


\section {Test Specifications}
\subsection{Primary goals:}
\begin{itemize}
    \item
          To test this, the drone will be weighed with a scale of a precision on 0,1 grams.
    \item
          In order to test the flight time, a stopwatch will be started from the moment the drone leaves the ground and is stopped as soon as it lands.
    \item
          This goal will be the tested trough FEM, ensuring that the chosen material for the drones body, will not rupture.
\end{itemize}

\subsection{Secondary goals}
\begin{itemize}
    \item
          This will be tested with the same method as primary goals tests point 2.
    \item
          This will be tested with a mobile phone, recording the drones landing, using the drones position compared to the timestamp of the video.
    \item
          This will be tested with the same goals as primary goals test point 3.
    \item
          This will be tested by the possibility of sending wireless signals to the drone, with the drone reacting to those send signals.
    \item
          This will be tested purely by ear, listening to the drones output.
    \item
          This will be tested by mobile phone video, looking at the drones positions at given timestamps.
    \item
          This will be tested trough summing the price for each single part, ensuring that it doesn’t exceed 500 DKK.
\end{itemize}

\subsection{Constraints}
\begin{itemize}
    \item
          This will be done with the same method as the secondary goals test, though ensuring the project cost is over 2000 DKK.
    \item
          To evaluate the time constraint point of the project, the goal fullfilments will be evaluated in the end of the project period. In the case that all primary goals are fulfilled, the constraint is succeeded.
    \item
          This will be tested with a stopwatch, ensuring that the hover time is atleast 5 seconds.
    \item
          This will be tested with making the drone take off right after a landing, making sure that the drone is fully operational at the second take-off.
    \item
          This will fulfilled by not employing any of the aforementioned in the drone.
\end{itemize}