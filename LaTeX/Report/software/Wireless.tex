\subsection{Wireless communication}
When prototyping, quadrotors are able to easily disconnect their wired connection. Therefore if mid-air communication is required, it is necessary to have wireless communication.

Commonly used communication for drones are primarily various forms of 2.4 GHz, as it provides the best middle ground between reach and speed. The 2.4 GHz has wide range of technologies supporting it, for example Bluetooth, Wi-Fi and RF communication protocols. 
If a high performance drone was the goal, RF communication would be chosen, but for the purpose of this project Bluetooth or rather BLE is optimal, as the microchip has it inbuilt. BLE stands for Bluetooth low energy and is a separate technology of Bluetooth and requires different parts. BLE has lower power requirements, as it is primarly one way communication, where the packets are not being send constantly, but only when needed or requested.

The implementation of BLE for nRF52840 in Arduino IDE usually relies on either Bluefruit LE or Adafruit BLE library, depending on whether pure C or Mbed is used.
For a device to use BLE, it needs to be either peripheral, which advertises itself or central, which connects to the advertised peripheral. Peripheral has services, for example control service, which can have characteristics like state with 0 and 1 for turning on and off.
This characteristic can have different options. Write, where the central writes to this characteristic, read if central reads from it and notify which lets the central know, that it changed. Also a combination of these can define each characteristic.

In this case, the drone was chosen as a peripheral with characteristics mentioned above. To it, a controller with a joystick and two buttons connects as a central and writes to the characteristics. This was not used in the project, as the project did not reach stages requiring this. 
Instead, there is kill-switch. This was made for first free-flight testing. It contains only one characteristic, which switches between 0 and 1 no matter what it receives which starts the drone on 1 and stops on 0. It can be controlled from a mobile phone for the sake of reliability. The option of sending gyro-data from a phone to the drone to mimic a controller was also explored, but required packet management, as the data packets send were too long.

Last option which was explored was logging data through BLE. The challenging part is that BLE is aimed mainly at mobile device and proprietary applications. Therefore there are no available applications for Windows which allow reading data from BLE devices and mobile phone applications do not offer the option of saving larger amounts of data. As a consequence of mentioned, data-logging through BLE was abandoned.